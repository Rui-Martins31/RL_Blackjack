\documentclass[12pt]{article}

\usepackage{geometry}

\geometry{a4paper,portrait,top=30mm,bottom=20mm,left=2.54cm,
	right=2.54cm,headsep=1mm,headheight=30mm,footskip=5mm}

\renewcommand{\familydefault}{\sfdefault}

\usepackage{sectsty}
\sectionfont{\fontsize{15}{18}\selectfont}
\subsectionfont{\fontsize{12}{15}\selectfont}

\usepackage[utf8]{inputenc} % accents i altres
\usepackage[T1]{fontenc}
\usepackage{xcolor}

\usepackage{multirow,tabularx}

\PassOptionsToPackage{hyphens}{url}
\usepackage[hidelinks]{hyperref}
\usepackage{graphicx}
\graphicspath{{Figures/}}

\newcommand{\course}{Dynamic Programming and Learning for Decision and Control}
\newcommand{\courseCode}{M.EEC026}
\newcommand{\lectiveYear}{2025--2026}
\newcommand{\semester}{1\textsuperscript{st}}
\newcommand{\degreeYear}{2\textsuperscript{nd}}

\usepackage{adjustbox}
\usepackage{lastpage}
%------------------------------ Header Footer ----------------------------------
\usepackage{fancyhdr}
	\pagestyle{fancyplain}					
	\fancyhead{}
	\fancyhead[]%
	{%
		\ifnum\thepage=1%
			\begin{minipage}{0.25\textwidth}
				{\includegraphics[width=0.9\textwidth]{images/logotipoUP.png}}
			\end{minipage}\hfill%
			\begin{minipage}{0.75\textwidth}
				\begin{flushleft}
					{\footnotesize\sc M.EEC $\vert$ \degreeYear{} Year}\\
					{\footnotesize\sc\courseCode{} $\vert$ \course{} $\vert$ \lectiveYear{} –- \semester{} Semester}
				\end{flushleft}
			\end{minipage}\\
			\rule[10pt]{\textwidth}{0.5pt}\\[-12pt]%
			\vspace*{1mm}
		\else
			\begin{adjustbox}{varwidth=\textwidth,raise=5mm}
				\centering
				\textbf{Intermediate Report}\\
				\rule[10pt]{\textwidth}{0.5pt}\\[-12pt]
			\end{adjustbox}
			\vspace*{-3mm}
		\fi%
	}
	\fancyfoot{}	
	\fancyfoot[CE,CO]%
		{%
			\footnotesize
			\rule{\textwidth}{0.5pt}\\[2mm]
			\thepage{}$/$\pageref{LastPage}%
		}
	\renewcommand{\headrulewidth}{0pt}					
	\renewcommand{\footrulewidth}{0pt}	
%-------------------------------------------------------------------------------

\usepackage{amstext,amsfonts,amsmath,amsthm,graphicx,amssymb,amscd,epsfig}
\usepackage{rotating}

\usepackage{subfig}
\captionsetup*[figure]{position=bottom}
\captionsetup*[subfigure]{position=bottom}


\usepackage{longtable}
\usepackage[font=small,labelfont=bf]{caption}
\usepackage{multirow}
\usepackage{float}
\usepackage{multicol}
\usepackage{wrapfig}

\usepackage[backend=biber, style=ieee]{biblatex}

% After page
\usepackage{afterpage}

% Listing - Python
\usepackage{listings}
\definecolor{codegray}{rgb}{0.5,0.5,0.5}
\definecolor{codebackground}{rgb}{0.95,0.95,0.92}
\lstset{
    language=Python, 
    basicstyle=\footnotesize\ttfamily, % Font size and family
    backgroundcolor=\color{codebackground}, % Background color
    commentstyle=\color{codegray}, % Comments style
    keywordstyle=\color{blue}\bfseries, % Keywords style
    numberstyle=\tiny\color{codegray}, % Line number style
    stringstyle=\color{red}, % String style
    frame=tb, % Frame on top and bottom
    tabsize=4, % Tab space width
    showtabs=false, % Show tabs as symbols
    showspaces=false, % Show spaces as symbols
    showstringspaces=false, % Show spaces in strings
    breaklines=true, % Automatic line breaking
    captionpos=b, % Caption position at bottom
    floatplacement=H, % Place listing exactly where specified
}

\begin{document}

%/---------- Title ---------/
\begin{center}
	\Large
	\textbf{Blackjack Report}
\end{center}
	
%/---------- Author ---------/
\begin{center}
	\large
	Author: 202104636 Rui Filipe Castro Martins
\end{center}

%/---------- Main text body ---------/
\section{Introduction} \label{section:introduction}

    Reinforcement Learning (RL) is a powerful way to train models without relying specifically in supervised learning. Based on the concept of action-reward, the agent (the one to be trained) is placed inside the environment where it can interact with it according to its own observations of the environment. From those observations, the agent can act. Actions can be either benefitial (the agent gets closer to the goal) or detrimental (the agent can end up "losing").
    
    In this project, the main goal is to design and train an agent to successfully learn how to play a game of blackjack. However, it is worth noticing that this is not a traditional blackjack; this one is called Easy21 and is based on a set of rules that sets it apart from the original game played in casinos.

% ----
\section{Easy21} \label{section:easy21}

    Easy21 can be seen as a spinoff of the original Blackjack game, where, for instance, cards like aces and faces cannot be drawed and the probability of drawing a red card is smaller than the probability of drwaing a black one. Also, red card should be counted as negative values for the final sum while black card as positive.

    This simplified version allows the player to choose from two independent actions: HIT or STICK. The player, in a first moment, is showed two card: one corresponding to the initial card of the dealer, and one that belongs to himself. From this observation the player chooses wether he wants to gamble and receive another card by HITting or if he prefers to simply STICK to his current hand.

    The winner is found by summing the value of the cards in each hand and evaluating which one is greater or by checking if either the player or the dealer busted by going over 21 or below 1.

    There are four scenarios when it comes to the final result:
        \begin{itemize}
            \item Both player and dealer get the same value when summing their hands which means there is a DRAW;
            \item The player has a greater hand than the dealer and WINS;
            \item The dealer has a greater hand than the player and WINS;
            \item Either the player or the dealer busts and the other side WINS.
        \end{itemize}

    Given the nature of this game, it is necessary to create a custom environment class that handles all the logic behind it. The subsequent snippets of code, shown in Listing \ref{listing:class_easy21} highlight the most important parts of the codebase.

        \begin{lstlisting}[caption={Easy21 Environment Class Initialization}, label={listing:class_easy21}, float=h!]
class Easy21:
    def __init__(self):

        # Observations
        self.cards_dealer: list[int] = []   # First card: 1-10
        self.cards_player: list[int] = []   # Total sum: 1-21

        # Actions
        self.actions: list[str] = [STICK, HIT]    # Stick, Hit

        # Probabilities
        self.draw_values: tuple[int] = (1, 10) # If we draw a red card we have (-10, -1) instead
        self.prob_red: float    = 1/3
        self.prob_black: float  = 2/3
        \end{lstlisting}
    
    This custom environment allows us to quickly generate synthetic data to train the agent. 

	
	% \afterpage{
	% 	\begin{figure}[t]
	% 		\centering
	% 		\includegraphics[width=0.7\linewidth]{california_wildfire_statistics.png}
	% 		\caption{California's burnt area over the years \cite{frontline_CA_fire_map}.}
	% 		\label{fig:burnt_area_annually}
	% 	\end{figure}
	% }

% ----
\section{Agent}

    To better implement a modular agent that adapts to different scenarios, a custom class and script were developed to guarantee this characteristic. Despite being based on past implementations such as the one found under OpenAI's Gymnasium documentation, this agent was fully built from scratch with simplicity in mind.  

    Listing \ref{listing:class_agent} shows the internal structure of the agent.

        \begin{lstlisting}[caption={Agent Class Initialization}, label={listing:class_agent}, float=h!]
class Agent:
    def __init__(
            self,
            observation: tuple[int] = _config.SCENARIO_OBSERVATIONS,
            num_observations: tuple[int] = _config.SCENARIO_OBSERVATIONS_NUM,
            actions: tuple[int] = _config.SCENARIO_ACTIONS,
            num_actions: int = _config.SCENARIO_ACTIONS_NUM,
            alpha: float = _config.ALPHA,
            gamma: float = _config.GAMMA, 
            epsilon: float = _config.EPSILON,
            epsilon_decay_factor: float = _config.EPSILON_DECAY_FACTOR,
            epsilon_min: float = _config.EPSILON_MIN
        ):

        # Info
        self.prev_observation  = observation
        self.num_observations  = num_observations

        self.actions           = actions
        self.num_actions       = num_actions

        # Current state
        # Q-matrix shape: (num_actions, player_sum, dealer_card)
        self.q_matrix          = np.zeros((self.num_actions, *self.num_observations))
        self.is_training: bool = False

        # Training
        self.epsilon: float       = epsilon
        self.epsilon_decay_factor = epsilon_decay_factor
        self.epsilon_min: float   = epsilon_min
        self.lr: float            = alpha
        self.gamma: float         = gamma
        \end{lstlisting}

    As seen in Listing \ref{listing:class_agent}, a \_config file was created to quickly modify the parameters of both environment and agent without having to access them internally. Listing \ref{listing:config_file} briefly describes this \_config file.

        \begin{lstlisting}[caption={Configuration File}, label={listing:config_file}, float=h!]
# Scenario
SCENARIO_OBSERVATIONS: tuple[int]     = (0, 0)
SCENARIO_OBSERVATIONS_NUM: tuple[int] = (10+1, 10+1) #(21+1, 21+1)         # +1 is the offset since array starts at [0]
SCENARIO_ACTIONS: tuple[int]          = (0, 1)
SCENARIO_ACTIONS_NUM: int             = len(SCENARIO_ACTIONS)

# Agent
AGENT_TRAIN: bool = True

# Episode
if AGENT_TRAIN: 
    NUM_EPISODES: int = 1_000_000
else:
    NUM_EPISODES: int = 10_000

# Training
EPSILON: float              = 1.0
EPSILON_DECAY_FACTOR: float = 0.99
EPSILON_MIN: float          = 0.1
ALPHA: float                = 0.1    # Learning Rate
GAMMA: float                = 0.99   # Discount Factor

# Path
PATH_SAVE_IMAGES: str   = "model/images/"
PATH_SAVE_Q_MATRIX: str = "model/q_matrix/"
        \end{lstlisting}

    % \begin{figure}[h!]
    % 	\centering
    % 	\includegraphics[width=0.9\linewidth]{}
    % 	\caption{System's diagram.}
    % 	\label{fig:system_diagram}
    % \end{figure}


% ----
\section{Results}

    Heat Map:

        \begin{figure}[h!]
        	\centering
        	\includegraphics[width=0.9\linewidth]{../model/images/policy_heatmap.png}
        	\caption{Heat Map.}
        	\label{fig:heat_map}
        \end{figure}


    3D Map of the Value Function:

        \begin{figure}[h!]
        	\centering
        	\includegraphics[width=0.9\linewidth]{../model/images/value_function_3d.png}
        	\caption{Heat Map.}
        	\label{fig:heat_map}
        \end{figure}
	

% ----
\section{Conclusions}

	


% ----

\end{document}